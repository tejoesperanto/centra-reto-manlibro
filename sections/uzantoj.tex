\section{Uzantoj en Centra Reto}\label{sec:uzantoj}
Ĉiu administrado de uzantoj en Centra Reto okazas ĉe Administrado » Uzantoj.

\subsection{Enkonduko al grupoj}
Ĉiuj roloj kaj permesoj en Centra Reto apartenas al iu \textit{grupo}. Grupoj donas al siaj membroj aliron al kelkaj funkcioj de la sistemo kiel ekz. administraj taskoj kaj kontribuado al cirkuleroj. Publikaj grupoj ankaŭ videblas en la profilo de la koncerna aktivulo.

Kelkaj grupoj estas \textit{supergrupoj} aŭ \textit{gepatraj grupoj}. Tio signifas, ke membreco en unu el ``infanoj'' de tia grupo aŭtomate donas membrecon en la rilata supergrupo. Ekzemple ĉiuj Komitatanoj A aŭtomate membras en la grupo `Komitatano' ĉar ili membras en `Komitatano A'.

\subsection{Kreado de nova uzanto}
Oni ne povas mem aliĝi al Centra Reto, necesas unue ricevi inviton de iu administranto. Por sendi tian inviton, uzu la formularon ``Krei novan uzanton'' plej supre. Tie endas indiki la retpoŝtadreson de la invitonto. Tiu ĉi retpoŝtadreso estos publike videbla en ties profilo, kaj pro tio estas tre grave unue peti konsenton de la koncerna homo, pri tio ke tiu adreso estos publika.

Antaŭ ol vi kreas novan konton por iu persono, bonvolu unue kontroli sube ĉe ``Uzantoj en Centra Reto'' ĉu la koncerna homo jam havas konton. Se la homo antaŭe estis aktivulo en TEJO, poste forlasis la postenon, kaj nun denove ekaktivis, povas esti ke la homo jam havas malŝaltitan konton. Serĉu en la formularo sub ``Uzantoj en Centra Reto'' per la nomo de la aktivulo. Se venas ĝusta rezulto, alklaku la vicon kaj premu la butonon ``Ŝalti ensaluton''. Ne forgesu ankaŭ aldoni la uzanton al la ĝustaj grupoj.

\subsection{Forigo de uzanto}
Ne eblas forigi uzanton kiu jam aliĝis al Centra Reto pro arkivaj kialoj. Vi tamen povas malŝalti ties ensaluton, se la homo ne plu bezonas konton. Antaŭ ol vi faras tion, ĉiam memoru unue forigi la eksaktivulon el siaj antaŭaj grupoj.

\subsection{Ĝisdatigo de grupoj}
Estas tre grave ĉiam certigi ke la uzantodatumbazo en Centra Reto estas ĝisdata. Se iu komitatano forlasas sian postenon, venas nova estraro aŭ io simila, vi devas tuj ĝisdatigi tion en Centra Reto---alie la homo ankoraŭ havos aliron al la konfidencaj sistemoj en Centra Reto.

Por ĝisdatigi ies membrecon en iu grupo, trovu la homon en ``Uzantoj en Centra Reto'' uzante la serĉfunkcion, alklaku la vicon kaj ĝisdatigu la grupojn de la aktivulo. Se la aktivulo ne plu apartenas al ajna grupo, bone pripensu ĉu ri entute bezonas konton en Centra Reto. Se ne, vi povas malŝalti ties konton. Se poste la koncerna homo denove ekaktivas en TEJO ĉiam eblas reŝalti la konton.
